\documentclass{llncs}

% Package for EPS-graphics
\usepackage{graphicx}

\usepackage{times}
\usepackage{t1enc}

\usepackage{epsfig}
\usepackage{subfigure}
\usepackage{fancyvrb}

\usepackage{float}
\floatstyle{plain}
\newfloat{codesample}{htbp}{loco}
\floatname{codesample}{Code example}


\begin{document}

% Paper title
\title{
	The Palladio Component Meta Model: Towards an Engineering Approach to Software Architecture Design
}

\author {Ralf H. Reussner \and Steffen Becker \and Jens Happe \and Klaus Krogmann \\
	\email{{ralf.reussner|steffen.becker|\\ jens.happe|klaus.krogmann}@informatik.uni-oldenburg.de} 
}
\institute{
	Software Engineering Group, Department of Computing Science\\University of Oldenburg,  Germany \\
}
\maketitle			

\begin{abstract}
The Palladio Component Meta Model described in this article is a rich component meta model, specifically designed to support the analysis of Quality of Service (QoS) attributes and interoperability problems of a component-based software architecture. The meta model defines entities such as components, connection, and interfaces. A unique feature of this component meta model is the specification of the relationship between provided and required functionality and quality of service properties of a component. This concurs with the specification of components with parametric contracts, which take into account that the functional and quality properties of a component heavily depend on its context.
Besides, ease of extensibility and closeness to UML2 concepts were driving forces during the design of the Palladio Component Meta Model.  

jeweils ein satz zu:
- was ist besser als bei UML, CCM und Co.
- Bereits erfolgreich eingesetzt: Webserver modelliert und Antwortzeit bestimmt

\end{abstract}

% ############################
\section{Introduction}
%Motivation
- early analysis of a software system regarding its quality attributes (e.g., performance and reliability)
- quality-driven selection of architectural alternatives
- one Problem: degrees of freedom

CBSE as solution:
 - Restriction of these degrees of freedom
 - Use of components with well-known properties
 - Use of compositional prediction models which reflect the component-based architecture

- information beyond the "classical" box-and-line-diagrams required
-> ``rich components models" 

- Deal with typical problems of re-use:
-> detection of interoperability problems
-> component adaptation


%contribution
TODO: hier noch etwas genauer werden.
- modelling of the relationship between provided and required interfaces (realisation of parametric contracts)
- different considerations of a component depending on the view of the system: type, implementation, deployment, runtime (based on ...) 
   Motivation der Ebenen: QoS (Einfluss der Implementierung auf QoS)
- (formal) definition of the relationship between the layers
- Weiterentwicklung bestehender Component Meta Models (UML2, CCM)

Benefits:
 - prediction of component properties (functional (PC) and QoS) for components and system wide
 - discussion on components: helping clarify the term
 -> closeness to UML2
 - defined as UML stereotypes

%organisation
TODO. kommt wenn paper fertig.

%Problem Statement
\section{Scenarios of Component-Based Software Design}
- Ziel: was wollen wir mit den Beispielen erreichen?
	- Probleme bisheriger Modell aufzeigen
	- Einf�hrung der Ebenen motivieren.
	- durg�ngige Beispiele einf�hren
- sp�ter im Text: PCMM bereits erfolgreich eingesetzt wird, siehe Experiment von Anne.

\subsection{Example Scenarios}
\label{sec:examples}
%Introduction
- examples of common component based software architectures


\subsubsection*{Three Tier Architecture}
- Three Tier architecture, 3 Komponenten als Typ (Erst Komponenten dann Schnittstellen)
- Komponenten eingekauft bzw. bereits implementiert
- Deploytes 3-Tier architecture zur QoS Vorhersage


\subsubsection*{Web Server}
- Versch. Implementierungen eines Komponententyps, HTTP-Request Processor (Basic Component vs. Composite), Verweis auf UML 2.0 Komponentenbegriff
- Plugin-Konzept: Erst schnittstellen dann Komponenten (webserver plugins)

\subsection{Engineering Tasks to be Supported}
\label{sec:tasks}
Entwurf:
-> Top Down/Bottom Up Iterativ (Neuentwicklung, COTS-integration)
-> Interfaces first / Components first
Wartung:
-> �nderungsunterst�tzung
Qualit�t:
-> Vorhersagen von QoS Eignenschaften
-> Validation


%Existing Solutions and their criticism
\section{Problems of Current Component Meta Models}
-> UML2, CCM
	-> Keine/rudiment�re Protokolle
	-> Parm. Vertr�ge, SEFFs
	-> Analysierbare Qualit�tsannotation
	-> Trennung der Komponentenebenen
	-> Subtype nicht klar definiert
-> ADLs? -- Sidenote
- das Ganze mit Beispielen von oben veranschaulichen.

%Proposed Solution and why it is expected to be better
\section{Concepts of the Palladio Meta Model}
\label{sec:concepts}
\subsection{Terms as UML Stereotypes}
-> Protocol
-> Interface

\subsection{Abstraction Levels of Components}  
Type
- interfaces
	- signature list
	- protocol
- component-types
	- keine Unterschiedung zwischen composite and basic component
- provided relation (comp,int)
- required relation (comp,int)
 - nur wenn das Interface verwendet werden MUSS
   Bsp: CoR, Decorator, PlugIn (Grafikbibliothek)
 
Implementation
- das gleiche wie Type +
- basic components
 - Service Effect Specification (bc)
  - Signature List
  	- Parameter sind entweder Wertetypen oder Schnittstellen zu anderen Komponenten
  	  -> zus�tzliche Requires schnittstellen, Heiko: Zustand externer Komponenten
  - Protocol
- composite components
 - interne Struktur, ``Verkabelung''
- required relation (comp,int)
 - alle Abh�ngigkeiten, die durch die Implementierung erzeugt werden
 - Protocol aus SEFF oder interner Struktur
- evtl.: Implementierung der Architektur (System als Composite Component)
- modellierung von required interfaces und seff: detailierungsgrad 


Deployment
- Allocation von Komponenten auf HW und SW Resourcen
- deployed-on relation (comp, res)
- HW und SW Resourcen als Entit�ten
- Modellierung der Deploymentumgebung -> einfluss auf die Performance?
	- Zeitmessung bei den Basisoperatoren
	- interner Zustand ist Mini-SEFF auf Basisoperatoren (Schleifen und IFs selten??)



Runtime
??

\subsection{Relationships}
-> Association / Constrainsts: <Implements>, <Deploys>, <Executes>
   OCL, Zusatzattibute
- Co- und Contravarianz
- Subtype von Komponenten

\subsection{A Classification of Interface Models}
- CBSE 7 paper klassification von Schnittstellen modellen anhand der KLassifikation von Interoperarbilit�tsproblemen

\subsection{Parametric Contracts}
- contracts for components
- parametric contracts for components

\subsection{Roles of Interfaces}
- free floating interfaces
- provided and required interface as stereotypes (oben)
% computation of deployment specific interfaces

\subsection{Scenarios Revisited}

%Conditions and assumptions
\section{A Different Perspective}
- Soll ein Typ bereits Seff's enthalten?
- Welche Required-Interfaces sollen spezifiziert werden? Keine, Alle?
- Beziehung Typ -> Impl. Co- oder Contravariant?
- QoS Attribute f�r Typen?


\section{Conclusions}
- major contribution
- who will benefit
- newly open problems

\bibliographystyle{splncs}
\bibliography{palladio}


\end{document}
