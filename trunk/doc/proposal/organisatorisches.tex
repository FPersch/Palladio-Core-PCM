\section{Organisatorisches}
\label{sec:organisatorisches}

\subsection{Beteiligte Personen}
\label{sec:organisatorisches:per}
Das Projekt wird von Juniorprofessor Dr. Ralf Reussner und Dipl.-Wirtsch.-Inform. Steffen Becker betreut. Es ist ein regelm��iges Treffen mit mindestens einem der Betreuer alle ein bis zwei Wochen geplant. 

\subsection{Ben�tigte Resourcen}
\label{sec:organisatorisches:res}

F�r die Bearbeitung des Projektes werden folgende Resourcen ben�tigt, die zum Teil von der Abteilung, die das Individuelle Projekt betreut, gestellt werden. 

\begin{itemize}
\item Zur Datensicherung und Versionsverwaltung steht ein Server mit einem CVS Zugang zur Verf�gung. Auf diesem Server k�nnen beliebige Dokumente und der Quellcode des Projektes gespeichert werden. 

\item Es wird eine Klassenbibliothek zur Verf�gung gestellt, welche zur Modellierung der statische Kontrollflu�strukturen von Komponenten dient.

\item Der BSCW Server (Basic Support for Cooperative Work), eine Art Groupware Server zur Kommunikation und abteilungsinternen Ver�ffentlichung von Dokumenten, existiert. Ein bereits eingerichteter Account steht zur Verf�gung.

\item Zur Entwicklung des Frameworks wird die Programmiersprache C\# verwendet. Die Entwicklung erfolgt unter Verwendung der Entwicklungsumgebung {\em Mircosoft Visual Studio .NET 7.1} mit dem {\em Microsoft .NET Framework 1.1}.

\item Zur Verifikation des Quellcodes und als Debug-Hilfe werden die Tools .NUNIT und .log4Net verwendet.

\item Die Dokumentation des Quellcodes wird unter Verwendung von .NDOC aus den Kommentaren generiert.

\item Alle Entwicklungen und Tests werden auf einem privaten PC mit dem Betriebssystem {\em Microsoft XP Professional} ausgef�hrt. 
\end{itemize}

\subsection{Produkte}
\label{sec:organisatorisches:prod}

Im Laufe des Projektes werden die hier aufgelisteten und kurz erl�uterten Produkte entstehen.

\begin{itemize}
	\item {\em Proposal}, die einleitende Dokumentation des Projektes
	\item {\em Ausarbeitung}, die komplette Ausarbeitung des Projektes
	\item {\em Framework}, die Implementierung des entwickelten Frameworks
	\item {\em Vortragsfolien}, es wird einen kurzen einleitenden Vortrag vor dem Projekt und einen zusammenfassenden nach dem Projekt geben.
\end{itemize}

\subsection{Zeitliche Planung}
\label{sec:organisatorisches:plan}

Das Individuelle Projekt ist f�r einen Zeitraum von vier Monaten (17 Wochen) angesetzt. Hierbei steht vor Antritt des Projektes das Proposal bereits zur Verf�gung und das Projekt wurde im Rahmen eines Seminartreffens kurz vorgestellt. Es folgt nun der grobe zeitliche Plan des Projektes.

\begin{tabular}[t] {||c|c|l||}
\hline \hline
\textbf{Wochen} & \textbf{Produkt} & \textbf{Beschreibung}\\
\hline
1 & 								 & genauere Erfassung der Problematik\\
	&									 & und Sichtung der Klassenbibliothek\\
\hline
4 & erstes Inkrement & Entwicklung der Basisfunktionalit�t\\
\hline
1 & erstes Inkrement & Entwicklung eines ersten Prototyps\\
\hline
5 & zweites Inkrement & Erweiterung um dynamische Aspekte\\
\hline
3 & drittes Inkrement & Entwurf einer GUI zum Auswerten \\
  &                   & der gesammelten Daten\\
\hline
3 & Dokumentation & Zusammenstellung der Dokumentations-\\
	&								&	fragmente, die w�hrend der drei\\
	&								&	Inkremente entstanden sind\\
\hline \hline
\end{tabular}
 

