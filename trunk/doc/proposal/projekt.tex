\section{Ziel des Individuellen Projekts}
\label{sec:ziel}

Da, wie in der Einleitung (siehe Kapitel \ref{sec:einleitung:simvsmath}) verdeutlicht, die exakte Berechnung eines Komponentennetzwerks zu komplex oder gar unm�glich ist, soll im Rahmen dieses Projektes eine Simulationsumgebung zur Analyse und Auswertung einer bestimmten Konfiguration von Komponenten entstehen. Hierzu setzt sich das Projekt die Implementierung der Simulationsumgebung und die Bereitstellung der Infrastruktur zur Nutzung dieser Umgebung in Form eines Frameworks zum Ziel.
\par
Die Erstellung des Frameworks teilt sich in mehrere Inkremente, die sequentiell entworfen, entwickelt und getestet werden sollen. Der Inhalt dieser Inkremente ist im Folgenden beschrieben.

\subsection{Entwicklung der Basisfunktionalit�t}
\label{sec:ziel:eins}

Das erste Inkrement befasst sich mit der Erstellung der Basisfunktionalit�t der Simulationsumgebung. Hierzu geh�rt die Entwicklung von Strukturen zur Repr�sentation des Komponentennetzwerkes unter Integration der Klassenbibliothek zur statischen Modulierung des Kontrollflusses einer Komponente (siehe Kapitel  \ref{sec:organisatorisches:res}). In diese Strukturen werden zun�chst statisch Verz�gerungszeiten der Dienste der Komponenten und der verwendeten Konnektoren eingearbeitet. Diese werden dann im zweiten Inkrement um gewisse dynamische Aspekte erweitert (siehe Kapitel \ref{sec:ziel:zwei}).\\
Um die Strukturen zu einem Netzwerk zusammenzusetzen, ben�tigt das Framework einige Komponenten, deren Entwicklung ebenfalls in dieses Inkrement f�llt.
\par
Weiterhin werden in diesem Abschnitt des Projektes die Datenstrukturen, welche als Nachrichten die Komponenten durchlaufen und Informationen (z.B. das Eintreffen oder Verlassen eines Dienstes) sammeln, entworfen. Diese werden in einem Pool gesammelt und k�nnen nach der Simulation ausgewertet werden. Methoden zur Auswertung der Daten werden zum Teil im dritten Inkrement entwickelt.
\par
Zur Synchronisation mehrerer Nachrichten wird eine Uhr ben�tigt, die somit erforderlich f�r die Basisfunktionalit�t ist und daher ebenfalls in das erste Inkrement f�llt. 

\subsection{Erweiterung um dynamische Aspekte}
\label{sec:ziel:zwei}
Nachdem das erste Inkrement abgeschlossen und das Ergebnis ausreichend getestet wurde, befasst sich das zweite Inkrement mit der Erweiterung um dynamische Aspekte.
\par
Hierzu geh�rt haupts�chlich die dynamische Anpassung der Verz�gerungszeiten der Dienste der Komponenten. Diese kann in Abh�ngigkeit von der Anzahl gleichzeitig eintreffender Nachrichten (bei Single-Threaded-Kom\-po\-nen\-ten) oder der Aufteilung der Komponenten auf verschiedene Prozessoren geschehen. Weiterhin w�re die Anpassung der Zeiten der Konnektoren in Abh�ngigkeit der zu verarbeitenden Nachrichten denkbar.
\par
Verzweigungen im Kontrollfluss einer Komponente werden im ersten Inkrement mit einer statischen Wahrscheinlichkeit aufgel�st. Dies kann zu Problemen f�hren, wenn der Kontrollfluss durch die Verzweigung zur�ckgef�hrt wird und somit Rekursion entstehen kann. Hier gilt es nun eine dynamische Anpassungsf�higkeit in Abh�ngigkeit der Rekursionstiefe einer Nachricht zu erm�glichen. So lie�e sich beispielsweise eine Schleife modulieren, welche bei den ersten 9 Durchl�ufen mit einer Wahrscheinlichkeit von 1.0 zu dem Dienst vor der Verzweigung zur�ckspringt. Erst beim 10 Durchlauf der Verzweigung wird diese Verlassen. Zur Umsetzung wird ein Stack ben�tigt, welcher den Weg der Nachricht protokolliert.
\par
Ein weiterer entscheidender Entwicklungspunkt dieses Inkrements besteht in der M�glichkeit, von Diensten einer Komponente neue Nachrichten ins Netzwerk zu senden. Durch diese Modulierung lassen sich Threads simulieren, die von Diensten zur Abarbeitung einer bestimmten Aufgabe erzeugt werden.

\subsection{Auswertung des Datenpools}
Nach Vollendung des zweiten Inkrements steht bereits eine funktionsf�hige Simulationsumgebung mit ausreichender Dynamik zur Verf�gung. Die Komponenten, Konnektoren und das Netzwerk k�nnen durch den Nutzer spezifiziert werden. Weiterhin besteht die M�glichkeit, Nachrichten an einen Systemdienst zu senden und deren Weg zu protokollieren.
\par
Die gesammelten Daten liegen nach der Simulation in Form einer Datei vor, deren Auswertung jedoch noch unflexibel und un�bersichtlich ist. Hier kommt das dritte Inkrement ins Spiel. Es soll nun eine GUI entworfen werden, die die Auswertung der Daten erleichtert. Hierf�r ist sowohl eine angemessene Pr�sentation der Daten als auch eine M�glichkeit der Anfrage des Benutzers vorgesehen.
\\
\\
An dieser Stelle endet die Erstellung des Frameworks im Rahmen des Individuellen Projektes. Der folgende Abschnitt enth�lt einige Vorschl�ge �ber Erweiterungsm�glichkeiten der Simulationsumgebungen.

\subsection{Erweiterungsm�glichkeiten ausserhalb des Projektes}
\label{sec:ziel:erw}
Bis auf die Auswertung der gesammelten Daten bietet das Framework an dieser Stelle kaum Bedienkomfort. Eine sinnvolle Erweiterung w�re hier die Entwicklung einer GUI. Diese kann beispielsweise die Verwaltung der Komponenten und Konnektoren (Erstellen und Speichern) �bernehmen. Weiterhin w�re eine graphische Anzeige und Erstellung des Komponentennetzwerkes denkbar.