\documentclass[10pt]{article}

\usepackage[a4paper,left=2cm,right=2cm,top=2.25cm,bottom=2.25cm,headheight=0.0pt,headsep=0.0cm,footskip=0.5cm]{geometry}
%---- Deutsche Pakete -------%
\usepackage {ngerman}
\usepackage[latin1]{inputenc}
\usepackage{umlaut}

%----- Pakete, um mathematische Ausdr�cke zu drucken ---%
\usepackage{amsmath}
\usepackage{amssymb}

%----- Graphik-Pakete ------%
\usepackage{graphicx}
%\usepackage[pdftex]{graphics}

\usepackage{enumerate}
%\usepackage{hyperref}

\usepackage[colorlinks=true, pdfstartview=FitH, linkcolor=black,
citecolor=blue, urlcolor=blue]{hyperref}

\begin{document}
\begin{center}
{\Huge 2. Protokoll}\\
{\large der PG Ride.NET}
\end{center}
\ \\ \ \\
\fbox{
\begin{minipage}[b]{17.5cm}
	\begin{description}
	\item [Datum:] 01.11.2004
	\item [Protokollant:] Klaus Krogmann
	\item [Anwesende:] Holger Cremer, Daniela Feldkamp, Thilo Focke, Markus Fromme, Stefan Gudenkauf, Klaus Krogmann, Daniel M�ller, Frank Stransky, Rolf Streng, Yvette Telken, Oliver Trella, J�rgen Ulbts
	\item [Abwesende:] -
	\end{description}
\end{minipage}}
\\ \ \\ \ \\




%--- Tagesordnungspunkte ---%
\underline{Tagesordungspunkte}: \\
\begin{itemize}

\item Beginn der Sitzung: 16:05 h


\item \textbf{Schl�sselausgabe}
	
Die Schl�ssel f�r den Zugang zu U 104 sind bisher noch nicht verf�gbar und werden in den n�chsten Tagen erwartet.
		
		
\item \textbf{Logins auf Erde}	
	
F�r den Zugang zu Erde sind die ARBI-Logins ab sofort g�ltig. Damit ist auch das Palladio-CVS erreichbar. Die bereits vorliegenden Protokoll-Dateien sollen daher aus dem BSCW in das CVS �bernommen werden.
	
	
\item \textbf{CVS-Struktur}
	
Als Modulname f�r das CVS (:ext:[Benutzername]@erde.informatik.uni-oldenburg.de:\\/home/palladio/pallad/cvsroot) wurde "`RideDotNet"' festgelegt. Darin soll zun�chst die folgende Struktur gelten:
		
	\begin{itemize}
		\item src
		\item doc
		
		\begin{itemize}
			\item Protokolle (Format: "`[YY]-[MM]-[DD]-Protokoll-[NR].tex"')
			\item Vorlagen
			\item Drafts
		\end{itemize}
	\end{itemize}
	
Die Dokumente zum Seminar sollen nicht eingecheckt werden. Zudem sollen keine compilierten Dokumente ins CVS eingecheckt werden.
	
	
\item \textbf{Regelm��iges internes Treffen}

F�r das interne Treffen lie� sich kein g�nstigerer Zeitpunkt finden, weshalb der Termin weiterhin montags um 16:00 h (s. t.) liegen wird.


\item \textbf{Anforderungsdefinition}

Der Ausgangspunkt zur Ermittlung der Anforderungen soll ein Interview sein.	Als Grundlage der Interviews mit Ralf Reussner und Steffen Becker zu den Anforderungen sollen die Vorlagen von Stefan Gudenkauf herangezogen werden, die er im Rahmen seines IPs erstellt hat, und ins BSCW einstellen wird. Insbesondere sollen funktionalen und nicht-funktionalen Anforderungen erfragt werden. Aus zus�tzlichen Fragen zu vorhandenen Komponenten und der Abgrenzung zu verwendenter bestehender Frameworks sowie m�glichen Testf�llen soll eine "`Liste zum Abhaken"' erstellt werden. Im Interview sollen die Befragten zun�chst frei vortragen, erst anschlie�end sollen die Fragen gestellt werden.


\item \textbf{Zeitplan}

Parallel zur Seminarphase wird die Phase der \textbf{Anforderungsdefinition} mit dem ersten Interview mit Steffen Becker und Ralf Reussner starten. Die Phase der Anforderungsdefinition endet mit einem Meilenstein am 17.12.04. Als Dokumente der Anforderungsdefiniton sollen entstehen:

\begin{itemize}
	\item Interviews
	\item Anwendungsfalldiagramme
	\item Anwendungsfallschablone
	\item Testkategorien und Testplan
	\item Pflichtenheft
	\item Schnittstellendefinition (insbesondere in Absprache mit Matthias Uflacker)
	\item Projektplan
\end{itemize}

Die \textbf{Entwurfsphase} wird auf den folgenden Zeitraum bis zum 28.01.05 festgelegt.



\item \textbf{Urlaub}

Als erste Urlaubsphase wird die Zeit zwischen dem 20.12.04 und dem 02.01.04 festgelegt.


\item \textbf{Webserver}

Derzeit sind installiert: Apache, MySQL, PHPMyAdmin. Bis sp�testens Mittwoch (03.11.04) soll ein Content Management System (CMS) installiert sein und die Homepage verf�gbar sein. Die URL der Webseite lautet: http://hurrikan.informatik.uni-oldenburg.de:8080. Daneben ist eine kostenlose Domain (etwas .de.vu) angedacht. Zur Webseitenadministrationsassistenz stehen J�rgen Ulbts und Rolf Streng Frank Stransky zur Seite. Ab sofort sollen verf�gbare CMS getestet werden.

Projektpl�ne sollen entweder als Export aus MS Projekt auf der Homepage als statische Seite eingef�gt werden oder direkt �ber das CMS eingebunden werden k�nnen.


\item \textbf{Protokolle}

Die Verteilung der Aufgabe des Protokollanten soll reihum geschehen. Auf das protokollierte Treffen folgt die Aufgabe, das n�chste Treffen zu moderieren. Dazu geh�rt auch das Versenden von m�glichen Tagesordnungspunkten per E-Mail an den Verteiler. Erg�nzungen zu Themenbereichen k�nnen dann ebenfalls �ber den Verteiler erfolgen.
		
		
\item \textbf{Protokolle}

Zur Verwaltung von Bugs soll ein System �hnlich Bugzilla verwendet werden. Hier soll eine Erprobung erfolgen.

	
\item Ende der Sitzung: 17:15 h

\end{itemize}
\end{document}

