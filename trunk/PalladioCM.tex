% Version history:
% $Log$
% Revision 1.2  2005/07/14 12:40:24  sbecker
% Fixed typos in the intro
% Added parts and authors of section 2
%
% Revision 1.1  2005/07/02 15:33:14  rr
% erste Fassung
%
\documentclass{llncs}

% Package for EPS-graphics
\usepackage{graphicx}

\usepackage{times}
\usepackage{t1enc}

\usepackage{epsfig}
\usepackage{subfigure}
\usepackage{fancyvrb}

\usepackage{float}
\floatstyle{plain}
\newfloat{codesample}{htbp}{loco}
\floatname{codesample}{Code example}


\begin{document}

% Paper title
\title{
	The Palladio Component Meta Model: Towards an Engineering Approach to Software Architecture Design
}

\author {Ralf H. Reussner \and Steffen Becker \and Jens Happe \and Klaus Krogmann \\
	\email{{ralf.reussner|steffen.becker|\\ jens.happe|klaus.krogmann}@informatik.uni-oldenburg.de} 
}
\institute{
	Software Engineering Group, Department of Computing Science\\University of Oldenburg,  Germany \\
}
\maketitle			

\begin{abstract}
The Palladio Component described in this article is such a rich component meta model, specifically designed to support performance and reliability predictions. The meta model defines entities such as components, connection, interfaces and ports. A unique feature of this component meta model is the specification of the relationship between provided and required functionality and quality of service properties of a component. This concurs with the specification of components with parametric contracts which take into account component's functional and quality properties heavily depend on the component's context.
Besides, ease of extensibility and closeness to UML2 concepts were driving forces of its design.  
\end{abstract}

% ############################
\section{Introduction}
Explicitly modelling software architectures is no means by itself. One benefit of explicitly modelled software architectures is the early analysis of a software system regarding its quality attributes (e.g., performance and reliability), in particular to support the quality-driven selection of architectural alternatives and thus supporting an engineering approach to software development. The analysis of an architecture to predict the quality attributes of its later implementation is inherently difficult, as many degrees of freedom exist to implement the architecture. The use of component based software development helps in three ways to increase the power of model-driven software quality prediction. 
\begin{description}
	\item[Restriction of these degrees of freedom] 
	\item[Use of components with well-known properties]
	\item[Use of compositional prediction models which reflect the component-based architecture]
\end{description}

However, an architectural analysis requires architectural information beyond the "classical" box-and-line-diagrams. This leads to so-called "rich components models" which include information necessary for tools performing architectural analyses. The Palladio Component described in this article is such a rich component model, specifically designed to support performance and reliability predictions. Besides the above mentioned benefit of using software component for system predictability, the use an component based software development approach also has to deal with problems specific to the re-use of components. One of these implications of component re-use is the need for component adaptation. In addition, ease of extensibility and closeness to UML2 concepts were driving forces of its design. 


%contribution
The model presented in this article differs from other components in several ways.
\begin{enumerate}
	\item Explicitly modelling of the relationship between provides and requires interfaces 
	\item A layered component models, which explicitly distinguished "component types", "component implementation instances", "component deployment instances" and "component run-time instances".   
\end{enumerate}
The contribution of the first of this specifica is the support for the prediction of component properties which depend on the component context (functional and quality properties) and the architectural reasoning on system-wide quality attributes in dependency of the inner components properties and their inter-connection given by the software architecture. The second specifica contributes to the discussion on the term component by explicitly modelling components at different instance-levels.  
%organisation
This paper is organised as follows. Some prerequisites on component modelling are discussed in section \ref{sec:genremarks}. Elementary modelling constructs are presented in section \ref{sec:bme}. Section \ref{sec:layeredModel} presents models for components on different levels of abstraction, such as component types and the various component instance levels. The static structure of the entire component meta model is summarised in section \ref{sec:staticStructure}.

\section{Scenarios of Component based Software Design}
%evtl. example of web-server (DA Jens)
% ---> Steffen
% * Three Tier architecture, 3 Komponenten als Typ (Erst Komponenten dann Schnittstellen)
% * Komponenten eingekauft bzw. bereits implementiert
% * Deploytes 3-Tier architecture zur QoS Vorhersage

% ---> Jens
% * Versch. Implementierungen eines Komponententyps, HTTP-Request Processor (Basic Component vs. Composite), Verweis auf UML 2.0 Komponentenbegriff
% * Plugin-Konzept: Erst schnittstellen dann Komponenten (webserver plugins)

\subsection{Scenarios to be Supported}
\subsection{Problems of Current Component Meta Models}
\section{Concepts of the Palladio Meta Model}
\label{sec:genremarks}
\subsection{Terms as UML Stereotypes}
\subsection{A Classification of Interface Models}
% CBSE 7 paper klassification von Schnittstellen modellen anhand der KLassifikation von Interoperarbilit�tsproblemen
\subsection{Parametric Contracts}
%to be taken from various papers :-)
%contracts for components
%parametric contracts for components
\subsection{Roles of Interfaces}
% free floating interfaces
% provided and required interface as stereotypes
% computation of deployment specific interfaces
\subsection{Abstraction Levels of Components}  
\subsection{Scenarios revisited}
\section{Conclusions}


\bibliographystyle{ieeetrans}
\bibliography{quellen}


\end{document}
