\section{Deployment}
The step of placing or installing software on the target systems. This includes configuration steps if necessary.

General understanding for software components: Placement of a component in an execution environment.

Our understanding:
The deployment of a software component defines its context.
1.	Assembling of components (Connections)
2.	Allocation of components on resources

Both steps can be done independently (two roles, different persons). Nowadays they are generally considered as deployment.

A component can be embedded into different contexts, since it is a unit of independent deployment. This can happen several times within the same system. Leaving type theory.

Within the same system, we have that the same component has different contexts, including the wiring, the mapping to resources and the containment.

Explicit modelling of the context. 
Two dimensions: Assembly and Allocation, Computed vs. Specified Values.

--Table with different context aspects--

Assembly and allocation can contain special configurations of a component.

Connections and their deployment: Today, either fixed connection between components or naming services. Both do not allow individual connection of components. So, components are no real units of independent deployment at the moment. Technical realisation required.

Composite Components can only be deployed as one piece.
Resource diagram and assembly diagram are specified independently. 

Communication nodes, which only model network structures.
