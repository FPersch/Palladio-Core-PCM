\thispagestyle{empty}

\vspace*{15\baselineskip}

F�nf Punkte des Abstracts %: http://sdqweb.ipd.uka.de/wiki/Tipps_zum_Schreiben_von_Konferenzpapieren
   
   1.  Eingrenzung des Forschungsbereichs (In welchem Themengebiet ist die Arbeit angesiedelt? Wie ist das Verh�ltnis zum    
   2. Beschreibung des Problems, das in dieser Arbeit gel�st werden soll (Was ist das Problem und warum ist es wichtig dies zu l�sen?)   
   
   3. M�ngel an existierenden Arbeiten bzgl. des Problems (Warum ist es ein Problem, obwohl sich schon andere mit dem gleichen Thema besch�ftigt haben?)
 
 4. Eigener L�sungsansatz (Welcher Ansatz wurde in dieser Arbeit verwendet, um das Problem zu l�sen? Was ist der Beitrag dieses Artikels?)   
 
 5. Art der Validierung + Ergebnisse (Wie wurde nachgewiesen, dass die Arbeit die versprochenen Verbesserung wirklich vollbringt (Fallstudie, Experiment, o.�.); Was waren die Ergebnisse der Validierung (idealerweise Prozentsatz der Verbesserung)?)


