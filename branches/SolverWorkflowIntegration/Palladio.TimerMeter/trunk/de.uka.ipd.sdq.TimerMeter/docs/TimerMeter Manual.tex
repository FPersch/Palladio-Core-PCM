\documentclass{entcs} 
\usepackage{prentcsmacro}
\usepackage[latin1]{inputenc}
\usepackage{times}
\usepackage{graphicx}
\usepackage{subfigure}
\usepackage{fancybox}
\usepackage{float}
\usepackage{listings}
%\usepackage{ngerman} %translates "Abstract" from the template into German
%\input pdfcolor.tex 
\sloppy
% The following is enclosed to allow easy detection of differences in
% ascii coding.
% Upper-case    A B C D E F G H I J K L M N O P Q R S T U V W X Y Z
% Lower-case    a b c d e f g h i j k l m n o p q r s t u v w x y z
% Digits        0 1 2 3 4 5 6 7 8 9
% Exclamation   !           Double quote "          Hash (number) #
% Dollar        $           Percent      %          Ampersand     &
% Acute accent  '           Left paren   (          Right paren   )
% Asterisk      *           Plus         +          Comma         ,
% Minus         -           Point        .          Solidus       /
% Colon         :           Semicolon    ;          Less than     <
% Equals        =3D           Greater than >          Question mark ?
% At            @           Left bracket [          Backslash     \
% Right bracket ]           Circumflex   ^          Underscore    _
% Grave accent  `           Left brace   {          Vertical bar  |
% Right brace   }           Tilde        ~

% A couple of exemplary definitions:

\newcommand{\Nat}{{\mathbb N}}
\newcommand{\Real}{{\mathbb R}}
\def\lastname{Kuperberg}
\begin{document}
\begin{frontmatter}
  \title{\textsc{TimerMeter User and Programmer Guide}: Quantifying Accuracy and Invocation Costs of\\Software Timers accessed from Java Bytecode} 
%  \title{\textsc{TimerMeter}: Quantifying Accuracy of\\Timer Methods for System Analysis} 
  \author{Michael Kuperberg
%  	\thanksref{ALL}
  	\thanksref{michael}
  }
%%  \address{Chair for Software Design and Quality\\ Universit�t Karlsruhe\\ Karlsruhe, Germany} 
%  \author{Martin Krogmann
%  	\thanksref{ALL}
%  	\thanksref{martin}
%  }
%%  \address{Chair for Software Design and Quality\\ Universit�t Karlsruhe\\ Karlsruhe, Germany} 
%  \author{Ralf Reussner
%  	\thanksref{ALL}
%  	\thanksref{ralf}
%  }
%  \address{Chair for Software Design and Quality\\ Universit�t Karlsruhe\\ Karlsruhe, Germany}
%  \thanks[ALL]{Thanks to everyone who should be thanked} 
%  \thanks[ALL]{Chair for Software Design and Quality, Universit�t Karlsruhe (TH), Karlsruhe, Germany} 
  \thanks[michael]{Chair for Software Design and Quality, Universit�t Karlsruhe (TH), Karlsruhe, Germany, Email:\href{mailto:mkuper@ipd.uka.de} {\texttt{\normalshape mkuper@ipd.uka.de}}} 
%  \thanks[martin]{Email:\href{mailto:martin.krogmann@ipd.uka.de} {\texttt{\normalshape martin.krogmann@ipd.uka.de}}} 
%  \thanks[ralf]{Email:\href{mailto:reussner@ipd.uka.de} {\texttt{\normalshape reussner@ipd.uka.de}}}
%%%%%%%%%%%%%%%%%%%%%%%%%%%%%%%%%%%%%%%%%%%%%%%%%%%
%%%%%%%%%%%%%%%%%%%%%%%%%%%%%%%%%%%%%%%%%%%%%%%%%%%
\begin{abstract} 
\textsc{TimerMeter} \cite{kuperberg2009a} is an novel approach for systematically obtaining the invocation cost and the accuracy (resolution) of timer methods. \cite{kuperberg2009a} describes the principles of \textsc{TimerMeter} and a case study which applied a Java implementation of it to different timer methods. This manual documents this Java implementation of TimerMeter and provides guidelines on extending it. It also describes implemented enhancements and extensions that were included into TimerMeter after \cite{kuperberg2009a} was submitted. Finally, it provides a list of known issues and planned development steps, as well as a FAQ list.
\end{abstract}
%%%%%%%%%%%%%%%%%%%%%%%%%%%%%%%%%%%%%%%%%%%%%%%%%%%
\begin{keyword}
  Timers, counters, timer methods, precision, accuracy, invocation costs, fine-granular measurements, clustering
\end{keyword}
\end{frontmatter}
%%%%%%%%%%%%%%%%%%%%%%%%%%%%%%%%%%%%%%%%%%%%%%%%%%%
%%%%%%%%%%%%%%%%%%%%%%%%%%%%%%%%%%%%%%%%%%%%%%%%%%%
\section{Introduction}% (2.0 pp incl. title+abstract)
\label{sec:Introduction}
%%%%%%%%%%%%%%%%%%%%%%%%%%%%%%%%%%%%%%%%%%%%%%%%%%%
%%%%%%%%%%%%%%%%%%%%%%%%%%%%%%%%%%%%%%%%%%%%%%%%%%%
Text
%%%%%%%%%%%%%%%%%%%%%%%%%%%%%%%%%%%%%%%%%%%%%%%%%%%
%%%%%%%%%%%%%%%%%%%%%%%%%%%%%%%%%%%%%%%%%%%%%%%%%%%
\section{Applying TimerMeter and Evaluating the Results}% (2.0 pp incl. title+abstract)
\label{sec:Applying}
%%%%%%%%%%%%%%%%%%%%%%%%%%%%%%%%%%%%%%%%%%%%%%%%%%%
%%%%%%%%%%%%%%%%%%%%%%%%%%%%%%%%%%%%%%%%%%%%%%%%%%%
Text
%%%%%%%%%%%%%%%%%%%%%%%%%%%%%%%%%%%%%%%%%%%%%%%%%%%
%%%%%%%%%%%%%%%%%%%%%%%%%%%%%%%%%%%%%%%%%%%%%%%%%%%
\section{Project Structure and Libraries}% (2.0 pp incl. title+abstract)
\label{sec:Structure}
%%%%%%%%%%%%%%%%%%%%%%%%%%%%%%%%%%%%%%%%%%%%%%%%%%%
%%%%%%%%%%%%%%%%%%%%%%%%%%%%%%%%%%%%%%%%%%%%%%%%%%%
Text
%%%%%%%%%%%%%%%%%%%%%%%%%%%%%%%%%%%%%%%%%%%%%%%%%%%
%%%%%%%%%%%%%%%%%%%%%%%%%%%%%%%%%%%%%%%%%%%%%%%%%%%
\section{Code Description and Documentation}% (2.0 pp incl. title+abstract)
\label{sec:Code}
%%%%%%%%%%%%%%%%%%%%%%%%%%%%%%%%%%%%%%%%%%%%%%%%%%%
%%%%%%%%%%%%%%%%%%%%%%%%%%%%%%%%%%%%%%%%%%%%%%%%%%%
Text
%%%%%%%%%%%%%%%%%%%%%%%%%%%%%%%%%%%%%%%%%%%%%%%%%%%
%%%%%%%%%%%%%%%%%%%%%%%%%%%%%%%%%%%%%%%%%%%%%%%%%%%
\section{Extending and Modifying TimerMeter}% (2.0 pp incl. title+abstract)
\label{sec:Extending}
%%%%%%%%%%%%%%%%%%%%%%%%%%%%%%%%%%%%%%%%%%%%%%%%%%%
%%%%%%%%%%%%%%%%%%%%%%%%%%%%%%%%%%%%%%%%%%%%%%%%%%%
Andere Timer; andere Plattformen; andere Sprachen
\subsection{In Java 5 und sp�ter}
\label{sec:Extending:subsec:Java}
\begin{enumerate}
	\item In the constructor, set \texttt{timerClassName}, \texttt{timerMethodName}, \texttt{timerMethodUnit} and \texttt{timerMethodIsStatic} 
	\begin{itemize}
		\item if the timer method has a non-time \texttt{timerMethodUnit}\footnote{It is planned to replace this field's current type (\texttt{java.lang.String}) with an \texttt{enum}} (e.g. "ticks" with a certain frequency), that frequency can be set using \texttt{setExternallyDeterminedTimerFrequency}
		\item they are declared but not initialised in the \texttt{AbstractTimerMeter} class
		\item note that the method name is not allowed to contain a dot ("."), a semicolon (";"), or a parenthesis ($~(~)~\{~\}~<~>~[~]$)
		\item the \texttt{timerClassName} should be a fully-qualified class name referring to an non-abstract class. If \texttt{timerMethodIsStatic} is true, \texttt{timerClassName} does not need to have a constructor; otherwise, either (i) at least one\footnote{TimerMeter takes the first parameterless constructor by default, as returned by the Java Reflection API; if another constructor must be taken, the \texttt{initialiseInvokableTimerMethod} method can be overwritten.} parameterless constructor of that class must be declared and accessible for TimerMeter or (ii) the \texttt{initialiseInvokableTimerMethod} must be overwritten\footnote{It remains to be tested whether run(String args[]) that is final in class AbstractTimerMeter will use the subclass'}. The \texttt{initialiseInvokableTimerMethod} method is called internally by TimerMeter and is implemented in the \texttt{AbstractTimerMeter} class.
	\end{itemize}
	\item \texttt{obtainMeasurementsUsingDirectInvocation} must be overwritten (it is abstract in \texttt{AbstractTimerMeter} class)
	\item \texttt{timerMethodToOverride} must be overwritten (it is called by \texttt{obtainMeasurementsUsingOverridenTimerMethod}, and it is abstract in \texttt{AbstractTimerMeter} class)
	\item if the timer method's resolution is higher than its invocation cost and if that invocation cost has been determined using a more accurate timer (cf. \texttt{java.lang.System.currentTimeMillis}, s. case study in \cite{kuperberg2009a}), it is possible to set that invocation cost a \texttt{double} value using \texttt{setExternallyDeterminedTimerInvocationCost}.
	\item if the timer method's vendor provides a built-in facility to query the timer method's accuracy or resolution cost, this values can be stored in \texttt{AbstractTimerMeter} subclass using the methods \texttt{setOfficialTimerMethodAccuracy} and \texttt{setOfficialTimerMethodInvocationCost}, so that the unit (stored in \texttt{timerMethodUnit}) confirms with this information.
\end{enumerate}


Using the command-line parameters passed to \texttt{run(String[] args)}, the execution of the algorithm can be controlled (w.r.t. number of measurements, clustering algorithm, etc.). The concrete subclasses of \texttt{AbstractTimerMeter} class are expected to implement \texttt{public static void main(String[] args)} by initialising their declaring class (a subclass of \texttt{AbstractTimerMeter}), and by invoking \texttt{run} on the initialised instance.

For modifying the algorithm itself, see the comments in the source code of the \texttt{AbstractTimerMeter} class.
%%%%%%%%%%%%%%%%%%%%%%%%%%%%%%%%%%%%%%%%%%%%%%%%%%%
%%%%%%%%%%%%%%%%%%%%%%%%%%%%%%%%%%%%%%%%%%%%%%%%%%%
\subsection{For Java SE Versions 1.4 and earlier}
\label{sec:Extending:subsec:Pre5}
%%%%%%%%%%%%%%%%%%%%%%%%%%%%%%%%%%%%%%%%%%%%%%%%%%%
%%%%%%%%%%%%%%%%%%%%%%%%%%%%%%%%%%%%%%%%%%%%%%%%%%%
RetroTranslator? Remove generics?
%%%%%%%%%%%%%%%%%%%%%%%%%%%%%%%%%%%%%%%%%%%%%%%%%%%
%%%%%%%%%%%%%%%%%%%%%%%%%%%%%%%%%%%%%%%%%%%%%%%%%%%
\subsection{For Java ME and EE}
\label{sec:Extending:subsec:OtherJavaEditions}
%%%%%%%%%%%%%%%%%%%%%%%%%%%%%%%%%%%%%%%%%%%%%%%%%%%
%%%%%%%%%%%%%%%%%%%%%%%%%%%%%%%%%%%%%%%%%%%%%%%%%%%
Speicherbedenken? API-Einschr�nkungen?
%%%%%%%%%%%%%%%%%%%%%%%%%%%%%%%%%%%%%%%%%%%%%%%%%%%
%%%%%%%%%%%%%%%%%%%%%%%%%%%%%%%%%%%%%%%%%%%%%%%%%%%
\subsection{In anderen Programmiersprachen mit Java Bytecode als Ziel}
\label{sec:Extending:subsec:OtherJavaLangs}
%%%%%%%%%%%%%%%%%%%%%%%%%%%%%%%%%%%%%%%%%%%%%%%%%%%
%%%%%%%%%%%%%%%%%%%%%%%%%%%%%%%%%%%%%%%%%%%%%%%%%%%
Ruby etc.? Jython?
%%%%%%%%%%%%%%%%%%%%%%%%%%%%%%%%%%%%%%%%%%%%%%%%%%%
%%%%%%%%%%%%%%%%%%%%%%%%%%%%%%%%%%%%%%%%%%%%%%%%%%%
\subsection{In anderen Programmiersprachen}
\label{sec:Extending:subsec:OtherLangs}
%%%%%%%%%%%%%%%%%%%%%%%%%%%%%%%%%%%%%%%%%%%%%%%%%%%
%%%%%%%%%%%%%%%%%%%%%%%%%%%%%%%%%%%%%%%%%%%%%%%%%%%
.NET? C/C++? (Windows-Timer wiederverwerten... :-))
%%%%%%%%%%%%%%%%%%%%%%%%%%%%%%%%%%%%%%%%%%%%%%%%%%%
%%%%%%%%%%%%%%%%%%%%%%%%%%%%%%%%%%%%%%%%%%%%%%%%%%%
\section{Assumptions and Limitations}% (2.0 pp incl. title+abstract)
\label{sec:AssLim}
%%%%%%%%%%%%%%%%%%%%%%%%%%%%%%%%%%%%%%%%%%%%%%%%%%%
%%%%%%%%%%%%%%%%%%%%%%%%%%%%%%%%%%%%%%%%%%%%%%%%%%%
Text
%%%%%%%%%%%%%%%%%%%%%%%%%%%%%%%%%%%%%%%%%%%%%%%%%%%
%%%%%%%%%%%%%%%%%%%%%%%%%%%%%%%%%%%%%%%%%%%%%%%%%%%
\section{Contributing and Future Development}% (2.0 pp incl. title+abstract)
\label{sec:Future}
%%%%%%%%%%%%%%%%%%%%%%%%%%%%%%%%%%%%%%%%%%%%%%%%%%%
%%%%%%%%%%%%%%%%%%%%%%%%%%%%%%%%%%%%%%%%%%%%%%%%%%%
Text
%%%%%%%%%%%%%%%%%%%%%%%%%%%%%%%%%%%%%%%%%%%%%%%%%%%
%%%%%%%%%%%%%%%%%%%%%%%%%%%%%%%%%%%%%%%%%%%%%%%%%%%
\bibliography{manual}
\bibliographystyle{plain}
\end{document}
