\section{Instanzierung des Modells}

Das Komponentenmodell ist durch die im Namensraum \verb+Palladio.ComponentModel+ definierte Klasse \verb+ComponentModelEnvironment+ gekapselt. Diese l��t sich unter Verwendung des parameterlosen Konstruktors instanzieren. Mehrfache Instanzen des Komponentmodells sind problemlos m�glich, da auf die Verwendung von Klassen gem�� dem Singleton-Pattern basierend auf statischen Klassenvariablen \cite{lit:gof} verzichtet wurde. Beim Entwurf von Erweiterungen, die direkt in das Komponentenmodell einflie�en, ist dieses Konzept beizubehalten, um Kompatibelit�tsprobleme zu vermeiden. Konzepte zur Synchronisation zweier Modelle sind in der aktuellen Version des Komponentmodells nicht vorgesehen. Bestehen Anforderungen dieser Art, so sind diese unter Verwendung der Benachrichtigungsmechanismen in Verbindung mit der Anfrage und Builder-Schicht zu realieren.

Nach der Instanzierung steht ein leeres Modell zur Verf�gung. Wahlweise kann unter Verwendung der Builder (vgl. Kapitel \ref{sec:builder}) ein neues Modell erstellt oder ein persistent gespeichertes geladen werden. Alle Schnittstellen zu den jeweiligen Schichten sind �ber die das Modell kapselnde Klasse \verb+ComponentModelEnvironment+ erreichbar.