\section{Suchanfragen an das Modell}

Die Suche von Entit�ten und die Anfrage von Beziehungen zwischen diesen ist eine der wichtigsten Vorraussetzungen zur sinnvollen Nutzung des Komponentenmodells. Aufbauend auf diesen lassen spezielle  Abfragemethoden in Form von Vergleichs- oder Navigationsoperationen  entwickeln.

Dieses Kapitel stellt die bereits entworfene Basisfunktionalit�t zur Bearbeitung von Anfragen an das Komponentenmodell vor. Es folgt anschlie�end die Pr�sentation zweier in Planung befindlicher spezieller Anfragemethoden.

\subsection{Allgemeine Anfragen}
\label{sec:query:allgemein}

Allgemeine Anfragen beschr�nken sich auf die Eigenschaften von und die Beziehungen zwischen den Entit�ten. Erstere sind hierbei in die drei Ebenen \emph{Type-Level}, \emph{Im\-ple\-men\-ta\-tion-Level} und \emph{Deployment-Level} (vgl. \cite{lit:cm}) unterteilt.

Alle Anfragemethoden sind �ber das Interfaces \verb+IQuery+ direkt aus der das gesamte Modell kapselnden Klasse \verb+ComponentModelEnvironment+ erreichbar. Dieses Interface er\-m�g\-licht gezielte Erfragung von Beziehungen zwischen Entit�ten pro Ebene. Die Entit�ten selber und damit auch ihre Attribute sind unter Verwendung der im Interface \verb+IQueryRepository+ definierten Methoden erreichbar. Hierbei ist der entsprechenden Methode die ID der Entit�t zu �bergeben. Kann die Entit�t nicht gefunden werden, so wird eine entsprechende Ausnahme ausgel�st.

Beziehungen zwischen den Entit�ten m�ssen �hnlich wie bei den Buildern (vgl. Kapitel \ref{sec:builder}) ausgehendend von einem Bezugspunkt gestellt werden. Besteht beispielsweise Interesse an allen Schnittstellen einer Komponente, so ist unter Verwendung der Methode \verb+QueryComponent+ die entsprechende Komponente zu befragen. Als Ergebnis liefern die Methoden je nach Art entweder eine einzige oder eine Liste von IDs zur�ck. Gem�� dem Prinzip der Trennung von Entit�ten und deren IDs (vgl. Kapitel \ref{sec:kern:beschr:ids}) werden nie die Entit�ten selber als Parameter verlangt oder als Ergebnis zur�ckgegeben.

Aufbauend hierauf lassen sich Anfragen auf h�herer Abstraktionsebene stellen. Hierzu geh�rt beispielsweise die Navigation im Modell oder der Vergleich von Entit�ten anhand verschiedener Kriterien. Hintergr�nde, Anforderungen und L�sungsideen dieser beiden in Planung befindlichen Bestandteile werden in den zwei folgenden Teilkapiteln vorgestellt. 


