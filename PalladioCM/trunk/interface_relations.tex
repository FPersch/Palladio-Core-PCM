%
% $Id$
%
% $Log$
% Revision 1.4  2005/11/07 18:49:39  sbecker
% Added first thoughts on provided and required interfaces
%
% Revision 1.3  2005/10/28 14:25:13  sliver
% *** empty log message ***
%
% Revision 1.2  2005/10/26 14:35:52  sbecker
% Started section on provided interfaces
%
%

\section{Interface Relations}

After introducing the interface concept in the preceding section, we introduce the relationships a component can have with an interface. For the reading of this section, consider an intuitive understanding of the concept of a software component. Components are discusses in detail in the following section, but we need the following concepts to justify our view on components.

According to Szyperski, Components have only explicit and contractually specified dependencies to their context. As interfaces serve as contracts it makes sense to use them as the contracts needed. An interface contract has to roles associated: supplier and requester. Thus, it is important whether the component takes the supplier or the requester role. We use the "`provides"' association to indicate the supplier role and the "`requires"' association to indicate the requester.

The question remains what is the semantics of the two relationships. For each of them there are several options, which we discuss in the following. 

\subsection{Interpretations of Provided Interfaces}
TODO: Discuss this with Ralf and Jens

\begin{enumerate}
\item All methods and protocol sequences in the interface specification are implemented by the component and hence \emph{can} be offered by the component during runtime. Whether the component offers the interface depends on the conditions available at runtime.
\item All methods and protocol sequences have to be supported by the providing component independent from any conditions. 
\end{enumerate}

The second definition is useful in the context of systems which are fully under control of its implementer. He can design the environment so that all dependencies are fulfilled. Hence, the interface contract can be guaranteed by design. With third pary deployable components the situation is different. The first interpretation seems to be more appropriate as the control over the runtime environment is with the assembler and not with the producer of the component.

\subsection{Interpretations of Required Interfaces}
We discuss three options for the semantics of required interfaces which we know about. Note that we distinguish them with numbers for further reference in later sections.

\begin{enumerate}
\item	A required interface can be used, but additional interfaces can be added.
\item	Only the listed required interfaces can be used.
\item	A required interface has to be used in a predefined way and certain call sequences have to be executed. 
\end{enumerate}

%TODO discuss these options
